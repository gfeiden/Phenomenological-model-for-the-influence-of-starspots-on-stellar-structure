\section{Introduction}
Late-type stars often exhibit characteristics and behavior that is consistent with the presence of brightness asymmetries near their optical surface. A well known example is a periodic sinusoidal brightness modulation observed in photometric lightcurves of a large number of stars (e.g., Lockwood et al. 2007; Ciardi et al. 2011). In addition, spectroscopic evidence exists supporting the idea of stars possessing a distribution of temperatures at their surface, including variations of individual spectral line profiles (e.g., Barnes & Collier Cameron 2001), the shape of molecular band features (e.g., O'Neal et al. 1998), and shapes of broader spectral energy distributions (e.g., Stauffer et al. 2003). These brightness asymmetries are thought to be the result of localized spots, analogous to sunspots, that are either darker or brighter than the star's ambient photosphere. Despite the near ubiquity of brightness asymmetries among stars with convective envelopes, relatively few studies have investigated their influence on stellar structure and evolution (Spruit 1982a; Spruit & Weiss 1986; Jackson & Jeffries 2014; Somers & Pinsonneault 2015).

\subsection{Theoretical Background}
On the Sun, spots are the manifestation of a concentration of (vertical, i.e., normal to the solar surface) magnetic flux \citep{Hale1908}. Spots appear dark because the plasma interior to the magnetic structure is cooler than the surrounding plasma in the photospheric layers. This produces a significant intensity contrast resulting in spots' dark appearance. How the confined plasma comes to be cooler than the surrounding photospheric layers is---and has been---a question of substantial interest. 

Suppression of convection from Lorentz forces generated as feedback in response to convective flows, thereby blocking warm material from reaching the solar surface, was an early proposal to explain the temperature within magnetic structures \citep{Biermann1941}. However, subsequent studies revealed that radiative loses could not fully explain the observed temperature within the magnetic structures, leading to the suggestion that convection is only \emph{inhibited} by magnetic fields \citep{Deinzer1965}. Concurrently, an alternative mechanism was proposed whereby cooling occurs through the dissipation of energy by hydromagnetic waves (Alfv\'{e}n waves) generated by plasma motions beneath and within magnetic structures \citep{Alfven1942,Parker1974b}. Although the latter mechanism required a substantial loss of energy through hydromagnetic waves, it found favor due to an observational lack of bright rings that are expected to exist around magnetic structures resulting from the diffusion of heat trapped below the structures emerging at the surface \citep{Parker1974b}. The absence of bright rings following the magnetic inhibition of convection was subsequently provided an explanation \citep{Spruit1977,Spruit1982a}. Now, the current consensus is that magnetically confined plasma cools due to radiative loses at the surface, while inhibition of convective flows below the magnetic region starves the plasma of warm material upwelling from deeper in the star \citep[see, e.g., reviews by][]{Rempel2011,Stein2012}.  

Magnetic fields oppose motions perpendicular to their lines of force through the generation of Lorentz forces, notably a magnetic tension force. Convective flows moving in a direction parallel to magnetic flux tubes are therefore not inhibited, but the ability of the flows to overturn is strongly opposed by magnetic tension forces. As a result, the efficiency with which convective flows transport energy is reduced. Material within the magnetic structure is therefore isolated from surrounding convective flows exterior to the magnetic structure, which cannot penetrate the magnetic boundary, leaving interior plasma to gradually cool through radiative loses from the solar surface \citep{Stein2012}. Precipitated by the radiative cooling, gravity collapses material in the magnetic structure, establishing a smaller density scale height so to maintain stability \citep{Parker1978}. This creates a pressure differential between the interior and exterior of the magnetic structure, leading to a compression the magnetic structure until the total interior pressure (magnetic + gas) balances the pressure acting on the boundary of the magnetic structure \citep{Parker1978,Spruit1979}.

To account for the whereabouts of heat trapped below a magnetic structure, \citet{Spruit1977} modeled a sunspot as a perfect insulator---a so-called ``thermal plug model.'' The thermal plug model correctly predicts that flux is efficiently trapped, creating a reservoir of excess heat, much like is expected with magnetic structures. \citet{Spruit1977} proceeds to demonstrate that turbulent diffusion would then transport heat away from the reservoir, with radiation playing a negligible role owing to a weak lateral temperature gradient. Convective motions continue distribute heat from the reservoir throughout the Sun's convection zone, ultimately leading to a long term storage of excess heat.

However, the theory is not without challenges or uncertainties. For example, the physical mechanism by which magnetic flux is brought into the upper layers and then compressed into concentrated structures is not fully understood. The prevailing theory is that buoyantly unstable magnetic flux rises through the convection zone, distorted by convective flows, with upwelling material tending to pull tubes toward the surface, and strong convective downflows stretching thin sections of tube in the opposite direction. This theory carries with it the assumption that magnetic flux has its origins deep within the convection zone, near the solar tachocline. Yet, it may be possible for a near-surface distributed dynamo to generate magnetic flux \citep{Brandenburg2005b} whereby magnetic instabilities ensue in the solar photospheric layers, driving the concentration of magnetic flux into shallow structures \citep[the so-called negative effective magnetic pressure instability][]{Brandenburg2011}.

The blocked flux is a concern for stellar interior structure and evolution. Flux that is temporarily blocked, but quickly released, is of little consequence for the overall structure of the star. However, the blockage of flux over sufficiently long timescales will create feedback on the overall thermal structure of the star, which is a natural consequence of the assumption that stars are in thermal equilibrium. 

Multiple timescales are involved in the eventual redistribution of heat and the gradual feedback on a star's overall thermal structure. Timescale for turbulent diffusion, or that required for heat to be extracted from the reservoir, followed by the thermal timescale of the convective envelope, which is finally proceeded by a period of thermal restructuring \citep{Spruit1982a}. Meanwhile, the impact on stellar structure is unique to each phase.

Predictions from the thermal plug model are consistent with observations of sunspots, bolstering the viability of the model. If trapped heat were able to efficiently flow around its oppressor, one would expect to observe bright rings surrounding individual spots \citep{Spruit1982b}. This is, however, not observed with the exception of a brightening in a narrow wavelength region around 4300 Å \citep{Bray1964}. 

\subsection{Motivation}
Given potential uncertainties regarding the depth to which spots may reach into the stellar convection zone, we seek to develop a toy model for incorporating the influence of spots on stellar structure into stellar evolution models. Effects of spots can be two-fold: (1) they may redistribute emergent flux, altering observed colors and magnitudes of stars, and (2) they may, after a time, force a star to thermally restructure, as discussed above. 

The influence that any given spot or distribution of spots may have on a star was shown to be related to the timescale over which spots exists. However, there is an added consideration, which is the depth to which spots extend into the convection zone. While a shallow spot suppress emergent energy in a similar manner to a deep-seated spot, the redistribution of flux trapped beneath either spot is dependent on its depth.